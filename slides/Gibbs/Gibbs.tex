\documentclass{beamer}

% for themes, etc.
\mode<presentation>
{ \usetheme{boxes} }
\setbeamertemplate{navigation symbols}{}
\usepackage{times}  % fonts are up to you
\usefonttheme{serif} 
\usepackage{graphicx, color,pgf}
\usepackage{multimedia}
%\usepackage{media9}
%\usepackage{lmodern}
\usepackage{amsmath,amsfonts,amsthm, amssymb}
\usepackage{bm}
\usepackage{animate}
\newcommand{\dif}[2]{\frac{{\rm d} #1}{{\rm d} #2}}
\newcommand{\ddif}[3]{\frac{{\rm d}^2 #1}{{\rm d} #2 {\rm d} #3}}
\newcommand{\ildif}[2]{{\rm d} #1/{{\rm d} #2 }}
\newcommand{\ilpdif}[2]{\partial #1/{\partial #2 }}
\newcommand{\pdif}[2]{\frac{\partial #1}{\partial #2}}
\newcommand{\pddif}[3]{\frac{\partial^2 #1}{\partial #2 \partial #3}}
\newcommand{\ilpddif}[3]{\partial^2 #1/{\partial #2 \partial #3}}
\newcommand{\comb}[2]{\left (\begin{array}{c}{#1}\\{#2}\end{array}\right )}
\newcommand{\perm}[2]{^{#1}{\rm P}_{#2}}
\newcommand{\gfrac}[2]{\mbox{$ { \textstyle{ \frac{#1}{#2} }\displaystyle}$}}
\newcommand{\defn}{\begin{quote}{\bf Definition. }}
\newcommand{\edefn}{\end{quote}}
\newcommand{\thm}{\begin{theorem}}
\newcommand{\ethm}{\end{theorem}}
\newcommand{\its}{^{\sf -T}} % transpose inverse
\newcommand{\fv}{\hat{\bm{\mu}}}
\newcommand{\X}{{\bf X}}
\newcommand{\Xt}{\X\ts}
\newcommand{\y}{{\bf y}}
\newcommand{\A}{{\bf A}}
\newcommand{\bp}{{\bm \beta}}
\newcommand{\bmat}[1]{\left [ \begin{array}{#1}}
\newcommand{\emat}{\end{array}\right ]}
\newcommand{\E}{\mathbb{E}}
\newcommand{\beq}{\begin{equation}}
\newcommand{\eeq}{\end{equation}}
\newcommand{\Gi}{{\bf G}^{-1}}
\newcommand{\B}{{\bf B}}
\newcommand{\Bt}{{\bf B}\ts}
\newcommand{\D}{{\bf D}}
\newcommand{\V}{{\bf V}}
\newcommand{\p}{{\bf P}}
\newcommand{\K}{{\bf K}}
\newcommand{\Uo}{{\bf U}_1}
\newcommand{\Tk}{{\bf T}_k}
\newcommand{\Tm}{{\bf T}_m}
\newcommand{\Tkm}{{\bf T}_{km}}
\newcommand{\grad }{\nabla}

\newcommand{\eps}[3]
{{\begin{center}
 \rotatebox{#1}{\scalebox{#2}{\includegraphics{#3}}}
 \end{center}}
}
\newcommand{\epstwo}[6]
{{\begin{center}
\rotatebox{#1}{\scalebox{#2}{\includegraphics{#3}}}
\rotatebox{#4}{\scalebox{#5}{\includegraphics{#6}}}
\end{center}}
}


\newcommand{\tr}{{\rm tr}}
\renewcommand*{\thefootnote}{\fnsymbol{footnote}}
\newcommand{\bd}[1]{\ensuremath{\mbox{\boldmath $#1$}}}
\newcommand{\ts}{^{\mbox{\sf \tiny T}}}
% these will be used later in the title page
\title{Gibbs sampling for Bayesian Inference}

\author{{\bf Simon Wood}, University of Edinburgh, U.K.}


\date{}


% have this if you'd like a recurring outline
%\AtBeginSection[]  % "Beamer, do the following at the start of every section"
%{
%\begin{frame}<beamer>
%\frametitle{Outline} % make a frame titled "Outline"
%\tableofcontents[currentsection]  % show TOC and highlight current section
%\end{frame}
%}
%\usetheme{Dresden}

\begin{document}

\begin{frame}
\titlepage
\end{frame}

\begin{frame}{One at a time updating}
\begin{itemize}
\item In MH sampling it can be difficult to find a distribution that makes good proposals for the whole parameter vector, $\bm \theta$.
\item But there is nothing in the MH scheme to prevent us using a proposal that updates the parameters in smaller blocks, or even one at a time, cycling through the elements of $\bm \theta$.
\item Such single element proposals are easier to tune.
\item Another interesting fact is that if our proposal distribution is $\pi({\bm \theta}|{\bf y})$ then the MH acceptance probability is exactly 1 (try it).
\item This observation is not very practical for all at once updating of $\bm\theta$, but it turns out to be very useful for one at a time updating, and is one way of motivating {\em Gibbs sampling}.  
\end{itemize}
\end{frame}

\begin{frame}{Gibbs sampling}
\begin{itemize}
\item Let ${\bm \theta}_{-i}$ denote the elements of $\bm \theta$ other than $\theta_i$.
\item Suppose we want a random draw from $\pi({\bm \theta}|{\bf y})$, and we already have a random draw, ${\bm \theta}_{-i}^*$, from $\pi({\bm \theta}_{-i}|{\bf y})$. 
\item All we need to do is generate $\theta_i^*$ from $\pi(\theta_i|{\bm \theta}_{-i}^*,{\bf y})$ and append this to ${\bm \theta}_{-i}^*$ to get ${\bm \theta}^* \sim \pi({\bm \theta}|{\bf y}) = \pi(\theta_i|{\bm \theta}_{-i},{\bf y})\pi({\bm \theta}_{-i}|{\bf y})$.
\item To get a slightly different draw from the same distribution we could discard $\theta_i^*$ and draw a fresh value in the same way.
\item We could repeatedly cycle through all elements of $\bm \theta$, dropping each and replacing with a draw from its conditional $\pi(\theta_i|{\bm \theta}_{-i}^*,{\bf y})$. 
\item This method is {\em Gibbs sampling}.
\item To use it we need to identify the conditional densities. 
\item When identification of a conditional is not possible, we can use a Metropolis Hastings step for that parameter, or more efficient one dimensional alternatives.
\end{itemize}
\end{frame}


\begin{frame}{A simple example}
\begin{itemize}
\item Consider $n$ observations from model $x_i \sim N(\mu, \sigma^2)$, with priors
\begin{itemize}
\item $\tau = 1/\sigma^2 \sim \text{gamma}(a,b)$, i.e. prior $\pi(\tau) = b^a \tau^{a-1} e^{-b\tau}/\Gamma(a)$ 
\item Independently, $\mu \sim N(c,d)$.\end{itemize}
\item The joint density of $\bf x$, $\mu$ and $\tau$ is
\begin{eqnarray*}
\pi({\bf x},\tau,\mu) &=& \pi({\bf x}|\tau,\mu)\pi(\tau)\pi(\mu) \\&\propto& \tau^{n/2} e^{-\sum_i \tau(x_i-\mu)^2/2} e^{-(\mu-c)^2/(2d)} \tau^{a-1} e^{-b\tau}
\end{eqnarray*}
\item By Bayes Theorem, the conditional for each parameter will be proportional to $\pi({\bf x},\tau,\mu)$. The trick is to recognize the conditional from its kernel - i.e from the terms actually involving the parameter.
\item e.g. $ \pi(\tau|{\bf x},\mu) \propto \tau^{n/2+a-1}e^{-\sum_i \tau(x_i-\mu)^2/2 - b\tau} $,  recognizable as a $\text{gamma}(n/2+a,\sum_i (x_i-\mu)^2/2 + b)$.
\item Similarly, but with a bit more effort, $$
\mu|{\bf x},\tau \sim N\{(dn\bar x\tau + c)/(dn\tau + 1), d/(dn\tau+1)\}
$$
\end{itemize} 

\end{frame}

\begin{frame}[fragile]{Checking the priors}
\begin{itemize}
\item Let's apply the preceding model to the {\tt nhtemp} data. This will involve choosing the constants $a$, $b$, $c$ and $d$ defining the priors.
\item We should look at the data, and check the priors\ldots
{\scriptsize \begin{verbatim}
x <- nhtemp ## just to save later typing
a <- 1.2; b <- .6 ## gamma prior shape and scale
c <- 50; d <- 100 ## normal prior mean and variance
## check data and priors...
par(mfrow=c(1,3),mar=c(4,4,1,1))
hist(x,main="")
tau <- seq(0,10,length=400)
plot(tau,dgamma(tau,a,b),type="l")
mu <- seq(0,100,length=400)
plot(mu,dnorm(mu,c,d^.5),type="l")
\end{verbatim}}
\end{itemize}
\vspace*{-.5cm}
\eps{-90}{.4}{check-prior.eps}
\end{frame}

\begin{frame}[fragile]{Gibbs sampling loop}
\begin{itemize}
\item Now we can run the Gibbs sampling loop\ldots

{\scriptsize \begin{verbatim}
ns <- 10000 ## number of samples
th <- matrix(0,2,ns) ## sample storage
mu <- 0;tau <- 0.1 ## initial states
n <- length(x) 
## store constants needed repeatedly...
dn <- d*n;dnx <- dn*mean(x) 

for (i in 1:ns) { ## Gibbs loop
  ## update mu | tau, x ... 
  mu <- rnorm(1,(dnx*tau+c)/(dn*tau+1),sqrt(d/(dn*tau+1)))
  ## update tau | mu, x ... 
  tau <- rgamma(1,n/2+a,sum((x-mu)^2)/2 + b)
  
  th[,i] <- c(mu,1/sqrt(tau)) ## store as mean and sd
}
\end{verbatim}}
\end{itemize}
\end{frame}

\begin{frame}[fragile]{Checking the chains}
\begin{itemize}
\item First look at the trace plots\ldots
{\scriptsize \begin{verbatim}
par(mfrow=c(2,1))
## see ?plotmath for adding maths to plots
plot(th[1,],type="l",ylab=expression(mu))
plot(th[2,],type="l",ylab=expression(sigma))
\end{verbatim}}
\vspace*{-.5cm}
\eps{-90}{.4}{nh-trace.eps}
\item Convergence appears to be very rapid, and mixing good.
\end{itemize}
\end{frame}

\begin{frame}[fragile]{Checking autocorrelation}
\begin{itemize}
\item Let's check the impression of rapid mixing with ACF plots\ldots 
{\scriptsize \begin{verbatim}
par(mfrow=c(1,2))
acf(th[1,]);acf(th[2,])
\end{verbatim}}
\eps{-90}{.4}{nh-acf.eps}
\item The samples appear to be effectively independent of each other - actually this behaviour is special to this example - parameters with high posterior correlation would give chains with higher autocorrelation. 
\end{itemize}
\end{frame}

\begin{frame}[fragile]{Did the data matter?}
\begin{itemize}
\item A common error is to believe that having updated your beliefs about the parameters using data, then {\em all} those beliefs are now validated by data. The Covid literature provides many examples.
\item To avoid this it is a good idea to compare priors to posteriors, to see what effect the data had, if any. {\scriptsize \begin{verbatim}
hist(th[1,],xlab=expression(mu),main="",probability=TRUE)
mu <- seq(50,52,length=100)
lines(mu,dnorm(mu,c,d^.5),col=2)
hist(1/th[2,]^2,xlab=expression(tau),main="",probability=TRUE)
tau <- seq(0.1,1.2,length=100) 
lines(tau,dgamma(tau,a,b),col=2)
\end{verbatim}}
\vspace*{-1cm}
\eps{-90}{.4}{nh-popr.eps}
\item The posteriors are very different to the priors, which have little influence - here the posteriors are determined by the data. 
\end{itemize}
\end{frame}

\begin{frame}[fragile]{Posterior means and CIs}
\begin{itemize}
\item Nothing is different here to in the MH sampling case (except that we only have 2 parameter in this example).
{\scriptsize \begin{verbatim}
> pm <- rowMeans(th)
> names(pm) <- c("mu","sigma")
> pm ## posterior mean
       mu     sigma 
51.158764  1.263793 
> 
> ci <- apply(th,1,quantile,probs=c(.025,.975))
> colnames(ci) <- c("mu","sigma")
> ci ## confidence intervals
            mu    sigma
2.5%  50.83651 1.064045
97.5% 51.48939 1.511449
\end{verbatim}}
\end{itemize}
\end{frame}



\end{document}

