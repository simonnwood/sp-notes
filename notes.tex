\documentclass[10pt] {article}
\usepackage{epsf}
\usepackage{latexsym}
\usepackage[T1]{fontenc}
\usepackage{makeidx,epsfig,lscape}
\usepackage{natbib}
\usepackage{rotating}
\usepackage{floatpag}
\rotfloatpagestyle{empty}
\usepackage{graphicx}
\usepackage{graphics}
\usepackage{epsf}
\usepackage{amssymb}
\usepackage{amsmath}
\usepackage{amsthm}
\usepackage{latexsym}
\usepackage{comment}
\usepackage{bm}
\usepackage{times}
\setlength{\textheight}{22.5cm}
\setlength{\textwidth}{16cm}
\setlength{\oddsidemargin}{-5mm}
\setlength{\topmargin}{-0.8cm}
\setlength{\evensidemargin}{-5mm}
\usepackage{listings}
\newcommand{\grad}{\nabla}
\newcommand{\tr}[1]{\text{tr}(#1)}
\newcommand{\bp}{{\vm \beta}}
\newcommand{\X}{{\vf X}}
\newcommand{\E}{E}
\newcommand{\vf}{\bf} %% vector type-setting
\newcommand{\vm}{\bm} %% vector type-setting
\newcommand{\ts}{^{\rm T}}
\newcommand{\its}{^{-\rm T}}
\newcommand{\bmat}[1]{\left [ \begin{array}{#1}}
\newcommand{\emat}{\end{array}\right ]}
\newcommand{\eps}[3]
{{\begin{center}
 \rotatebox{#1}{\scalebox{#2}{\includegraphics{#3}}}
 \end{center}}
}
\DeclareMathOperator*{\argmin}{argmin}
\DeclareMathOperator*{\argmax}{argmax} %% original {arg\,max} to space arg max
\lstset{language=R,
        basicstyle={\ttfamily\small},
        keywordstyle=,
        showstringspaces=false,
        columns=flexible}

%% Definitions
\newcommand {\hide}[1] {\typeout{ #1 }}
%Comment out to print all
%\newcommand {\hide}[1] {{\it #1 }}
%Comment out to hide some
\newcommand{\beq}{\begin{equation}}
\newcommand{\eeq}{\end{equation}}
\newcommand{\dif}[2]{\frac{{\rm d} #1}{{\rm d} #2}}
\newcommand{\ildif}[2]{{\rm d} #1/{{\rm d} #2 }}
\newcommand{\ilpdif}[2]{\partial #1/{\partial #2 }}
\newcommand{\pdif}[2]{\frac{\partial #1}{\partial #2}}
\newcommand{\pddif}[3]{\frac{\partial^2 #1}{\partial #2 \partial #3}}
\newcommand{\ilpddif}[3]{\partial^2 #1/{\partial #2 \partial #3}}
\newcommand{\comb}[2]{\left (\begin{array}{c}{#1}\\{#2}\end{array}\right )}
\newcommand{\gfrac}[2]{\mbox{$ { \textstyle{ \frac{#1}{#2} }\displaystyle}$}}
\newcommand{\R}{{\sf R }}
\theoremstyle{definition}
  \newtheorem*{definition}{Definition}
  \newtheorem*{example}{Example}
%\newcommand{\vm}{\bm}
% comment out next line unless double spacing needed
%\renewcommand{\baselinestretch}{2}

\newtheorem{theorem}{Theorem}


\begin {document}

\centerline{\huge \bf Statistical Programming}

\section{Software Requirements}

To complete this course you will need to get yourself a free github account and install the following software on your computer: 
\begin{itemize}
\item R or Rstudio, git, pandoc and JAGS.
\item R packages: ggplot2, rjags, rmarkdown, debug.
\item Only if you do not already have latex installed, install tinytex. 
\end{itemize}
It is assumed that you have the basic computer skills to do this. If you do not, you will need to spend some time online acquiring them (that level of basic skills is not part of this course). If you have difficulty, you can post questions on piazza for other students to answer. Please answer other student's questions on installation on piazza. 

In more detail.
\begin{itemize}
\item R is the free statistical programming language and environment that we will use. You can get a copy from the Comprehensive R Archive Network (CRAN)\\
\verb+https://cran.r-project.org/+\\
just follow the links under `Download and Install R'. 
\item Rstudio is an alternative front end for R, which many people prefer to use. It provides an R code editor, R session window, R graphics window and other information, all in a convenient `integrated environment'. If you prefer this to R as available from CRAN, you can get it from\\
\verb+https://www.rstudio.com/+\\
at the foot of the page, under `R studio desktop'. 
\item git is a version control system. It helps you to write code in teams, as you must do for this course, without breaking each others work. It also lets you keep a record of the changes you make, so that you can go back to earlier versions, if you need to. \\
\verb+https://www.git-scm.com/+\\
contains instructions for installation. {\tt linux} systems often have it installed already and something like \verb+sudo apt install git+ will install it if not. 
\item pandoc is document conversion software used by the {\tt rmarkdown} package. Installation instructions:\\
\verb+https://pandoc.org/installing.html+ 
\item JAGS stands for `Just Another Gibbs Sampler'. We will use it for programming Bayesian models and sampling from them. Downloads are available here:\\
\verb+https://sourceforge.net/projects/mcmc-jags/files/JAGS/4.x/+\\
but under linux something like \verb+sudo apt install jags+ can be used. Note that if you have a new apple/mac with an apple silicon processor (rather than intel), then you will need to install MacPorts from\\ 
\verb+https://www.macports.org/install.php+\\
then install JAGS as described here\\
\verb+https://ports.macports.org/port/jags/+
\end{itemize}

Once the above are installed, you will need to install some packages from the R session command line. R/Rstudio can install packaged in a local directory/folder for you, but I tend to start R as administrator/super user/root and then install them (e.g. on linux I would start R from the command line as \verb+sudo R+). Here are the packages you need:
\begin{itemize}
\item ggplot2 provides a nice alternative to built in R graphics. Install by typing\\
\verb+install.packages("ggplot2")+\\
at the R command line.
\item rmarkdown provides a neat way of documenting data analyses using R. Install using\\
 \verb+install.packages("rmarkdown")+
\item rjags lets you use JAGS directly from R. Install using\\
 \verb+install.packages("rjags")+
\item Finally we need a debugging package, to help you to find errors in your code, or understand how code is working by stepping through it. What is built into R and Rstudio is not great, and a much better option is Mark Bravington's debug package. This is not available from CRAN, but only from Mark's repository, so to install, you need to do this:
\begin{verbatim}
options(repos=c("https://markbravington.github.io/Rmvb-repo",
        getOption( "repos")))
install.packages("debug")
\end{verbatim}
\item rmarkdown requires a latex installation in order to allow you to produce pdf documents. So, {\em only if you do not already have latex installed}, install {\tt tinytex} with
\verb+install.packages("tinytex")+
then if you are {\em really sure} that you do not already have latex installed run 
\verb+tinytex::install_tinytex()+
WARNING: this command may break existing latex installations!
\end{itemize}
Once you have the software installed, you need to do one more thing. Get a free github account (if you do not already have one).  \verb+https://github.com/join+ is where to get one. 

For git and github, there is much useful information available at 
\begin{itemize}
\item \verb+https://guides.github.com/+
\item \verb+https://git-scm.com/docs/gittutorial+
\item \verb+https://training.github.com/+
\end{itemize}
It is a good idea to have a look at this to get an initial feeling for what github is about.

\subsection{Using a terminal window, choose a text editor}

This course involves programming - that is writing text instructions that get a computer to do something. Anyone who programmes rapidly discovers that writing text to get a computer to do something is often quicker and more convenient than clicking your way through a graphical interface. This sometimes includes for basic tasks such as file copying, or moving between directories/folders. 

Whatever operating system you are using you will be able to launch a `terminal window' for this purpose. Make sure that for your operating system you know how to do this, how to list the names of the files in a directory/folder, how to change from one directory/folder to another, how to delete a file, and how to close the terminal window. 

You will also need to edit text files containing the computer code you write. If you choose to use Rstudio, there is a built in editor, and you might choose to just use that. If you use plain R then you will need an external editor. Word is not suitable, as it will insert hidden characters in your code that will cause problems, so you instead should use something simpler (e.g. wordpad or notepad on windows). Make sure you know what is available on your computer and use that. I tend to use emacs, but this has more features than you actually need for this course. 

\section{git and github}

{\tt git} is a system that lets you track changes in code files and to manage those files when several people are working on them. In particular it lets you maintain a central {\em repository} of your code, which serves as the backed up master copy. The repository is just a folder/directory containing the master copy of your code. Several people can make working local copies of the repository (on their own computers) and work on the code, and git provides the tools to allow management of the master copy in a way that avoids or resolves conflicts. A conflict is when two people made contradictory changes to the code. 

\subsection{Setting up a repo on {\tt github}}

An easy way to set up a central repository is to use {\tt github}.  
\begin{itemize}
\item Login to your github account on {\tt github.com}, click on {\tt Repositories} and then click on {\tt New} to set up a new repository.
\item Follow the instructions. You probably want your repository to be private if it is for working on assessed coursework.
\end{itemize}
For group homework, one person sets up the repo and invites their team mates to collaborate on it. 
\begin{itemize}
\item Log on to github, nagivate to your repo, and click on {\tt Settings}.
\item Click on {\tt Manage Access} from the left hand menu, and then on {\tt Invite a collaborator} - you will need their github user name. 
\end{itemize}
You can add files directly to your repo, via the web interface, but we will shortly cover how to add files to the local copy of your repo, and then {\em push} them to github. 

\subsection{Using {\tt git}}

Mostly, you will interact with your github repo via the {\tt git} software, and your local copy of your github repo. You will use {\tt git} by typing commands in a terminal window. To check everything is working before using {\tt git} for the first time you might open a terminal window and check the {\tt git} version, a follows (\verb+$+ indicates the command prompt, not something you type):
\begin{verbatim}
$ git --version
\end{verbatim}
This gives the answer \verb+git version 2.25.1+ for me. Here we will cover only the most basic use of {\tt git} and {\tt github}: the minimum needed to collaborate on projects and keep your work backed up. You can find much more at:
\begin{itemize}
\item \verb+https://git-scm.com/docs/gittutorial+
\item \verb+https://guides.github.com/+
\item \verb+https://training.github.com/+
\end{itemize}

\subsubsection{Making a local working copy of the repo}

After creating your github repo you need to make a local copy of it on your computer. 
\begin{itemize}
\item From your {\tt github} repo page (at {\tt github.com}) click on {\tt Code} select {\tt Clone} and copy the {\tt htpps} address for the repository that appears. For example, for my repo containing these notes, the address is \verb+https://github.com/simonnwood/sp-notes.git+
\item In the terminal window on your local machine change directory ({\tt cd}) to the folder where you would like the local repo to be located. Then type {\tt git clone} followed by the address you copied. e.g. 
\begin{verbatim}
git clone https://github.com/simonnwood/sp-notes.git
\end{verbatim}
\end{itemize}
A local copy of the repo is then created on your computer. 

\subsubsection{Modifying work and synchronising with the github repo}

The local copy of your repo is just a directory/folder containing the files in your repo, and a hidden {\tt .git} subdirectory that {\tt git} uses to maintain an index of the files that it should keep track of, and the history of changes made. {\tt git} knows about all the files in the {\tt github} repo you cloned, but will only start keeping track of other files that you may add to your local repo when you tell it to do so (with \verb+git add+). 

Similarly {\tt git} does not keep a record of all the changes you make to your code as you work on it. Rather, it takes and stores a `snapshot' of the state of the files it is tracking only when you tell it to, using a {\tt git commit} command. 

Neither does {\tt git} automatically save all these snapshots to your master {\tt github} repo. That only happens when you tell it to using {\tt git push}. This is quite useful, you can keep a detailed record of the changes you make while working on code locally, without modifying the {\tt github} master repo until you are satisfied that the changes are complete and working.   



The most basic way to use {\tt git} and {\tt github} is to work on the local copy of your repo, and once you have made the changes you want to make, have tested them and are happy with them, {\em commit} them and {\em push} them to the master {\tt github} repo. 



\section{Data re-arrangement and tidy data}

In raw form many data sets are messy and have to be tidied up for statistical analysis. By tidying is meant that the data are re-arranged conveniently for analysis, not that the information in the data is in anyway modified unless an actual error is identified (no removal of outliers, for example). Since `data tidying' or `data re-arrangement' sound too boring and clerical for many enthusiasts for Modern Data Science, you will often hear terms like `data wrangling' and `data munging' used instead. These sound suitably rugged, as if the types engaged in them return home aching and covered in dust and mud after a day engaged in the honest toil of these muscular pursuits.

By tidy data is meant that data are arranged in a matrix so that:
\begin{enumerate}
\item Each column is a variable.
\item Each row is an observation (a unique combination of variables).
\end{enumerate}
and it is recognised that what counts as `an observation' can depend on the subset of variables of interest. For example,
this data frame is in tidy form if interest is in student's grades each week. 
\begin{verbatim}
name    d.o.b.  nationality week grade
George  11/2/01 GB          1    B
George  11/2/01 GB          2    C
George  11/2/01 GB          3    A
Anna    15/6/00 GB          1    A
Anna    15/6/00 GB          2    B
Anna    15/6/00 GB          3    A
.       .       .           .    .
\end{verbatim}
But if we are only interested in birth dates and nationality of students, then we would really want to simplify to
\begin{verbatim}
name    d.o.b.  nationality 
George  11/2/01 GB          
Anna    15/6/00 GB          
.       .       .           
\end{verbatim}
It is easy to get the second data frame form the first (\verb+marks+, say) in base R using
\begin{verbatim}
student <- unique(marks[,c("name","d.o.b.","nationality")])
\end{verbatim}
which has selected the named columns of interest, and then found the unique rows. If you want to refer each row in \verb+marks+ to its row in \verb+student+ then you can use 
\begin{verbatim}
library(mgcv)
student <- unique(marks[,c("name","d.o.b.","nationality")])
ind <- attr(student,"index") ## ind[i] is row of 'student' corresponding to row i of 'marks'
\end{verbatim}

At this point, you are hopefully wondering about replicate data, and the definition of `observation' as `unique combination of variables'? If I repeat an experiment under identical conditions, could I not occasionally get identical results, and hence two identical rows in a data frame that are different observations? Well yes, but only if you do not include a variable recording the `replicate number'. If you don't record the replicate number then you do not really have tidy data, as it is not possible to distinguish the common error of accidental replication of some data, from genuine replicates.



\end{document}
